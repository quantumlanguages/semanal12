% type
\documentclass{article}

% format
\usepackage[letterpaper, margin=1.5cm]{geometry}

% font
\renewcommand{\familydefault}{\sfdefault}

% extra math env
\usepackage{amsmath}

% extra math symbs
\usepackage{amssymb}

% code
\usepackage{minted}

% macros
\newcommand{\tx}[1]{\texttt{#1}}
\newcommand{\es}{$\square$}
\newcommand{\pop}[2]{ \tx{#1} \succ \tx{#2}}
\newcommand{\push}[2]{ \tx{#1} \prec \tx{#2}}
\newcommand{\kr}{\rightarrow_{\mathcal{K}} \quad}
\newcommand{\br}{\rightarrow_{\beta} \quad}
\newcommand{\krs}{\rightarrow_{\mathcal{K}}^{\star} \quad}

% header
\title{
    Lenguajes de Programación 2020-1\\
    Facultad de Ciencias UNAM\\
    Ejercicio Semanal 12
}

\author{
    Sandra del Mar Soto Corderi\\
    Edgar Quiroz Castañeda
}

\begin{document}
    
    \maketitle

    \begin{enumerate}
        \item Evalúe el resultado de la siguiente expresión

        \[
            \tx{(letcc k in 10 * (continue k 3)) + 2}
        \]

        En en caso donde las expresiones antes de evaluar ya son valores y no
        forman parte de una expresión \tx{continue}, se saltará directamente a
        la siguiente evaluación.

        \begin{align*}
            & \pop{\es}{(letcc[Nat](k.10 * continue(k, 3))) + 2} \\
            \kr & \pop{sum(\_ + 2)}{letcc[Nat](k.10 * continue(k, 3))} \\
            \kr & \pop{sum(\_ + 2)}{(10 * continue(k, 3))[k := cont(sum(\_ + 2))]} \\
            \br & \pop{sum(\_ + 2)}{10 * continue(cont(sum(\_ + 2)), 3)} \\
            \krs & \pop{mul(10, \_),sum(\_ + 2)}{continue(cont(sum(\_ + 2)), 3)} \\
            \kr & \pop{continue(\_, 3), mul(10, \_),sum(\_ + 2)}{cont(sum(\_ + 2))} \\
            \kr & \push{continue(\_, 3), mul(10, \_),sum(\_ + 2)}{cont(sum(\_ + 2))} \\
            \kr & \pop{continue(cont(sum(\_ + 2)), \_), mul(10, \_),sum(\_ + 2)}{3} \\
            \kr & \push{continue(cont(sum(\_ + 2)), \_), mul(10, \_),sum(\_ + 2)}{3} \\
            \kr & \push{sum(\_ + 2)}{3} \\
            \krs & \push{\es}{5}
        \end{align*}

        \item Transforme la siguiente función a CPS

        \begin{minted}{haskell}
            -- función filter
            filter _ [] = []
            filter f (x:xs) 
                | f x = x :xs'
                | otherwise xs'
              where xs' = filter f xs
        \end{minted}

        y utilice su definición para calcluar \tx{cpsfilter cpseven [0..10]} 
        paso a paso.
        
    \end{enumerate}
\end{document}